自从有了计算机,编程语言就开发了数千种。事实证明,所有编译器都必须解决相同的任务,并且编译器的实现最好根据这些任务进行结构化。总得来说,有三个组成部分。前端将源代码转换为中间表示(IR);中端在IR上执行转换,其目标是提高性能或减少代码的大小;后端将IR生成为机器码。LLVM核心库为所有主流平台提供了一个由复杂转换和后端组成的中端。

此外,LLVM核心库还定义了一个中间表示,用作中间端和后端的输入。这种设计的优点是,只需要关心编程语言的前端即可。

前端的输入是源代码,通常是一个文本文件。为了使它更有意义,前端首先标识语言的单词,例如数字和标识符,通常称为标记。这一步由词法分析器执行。其次,分析了符号构成的句法结构。所谓的解析器执行这个步骤,结果是生成了抽象语法树(AST)。

最后,前端需要由语义分析器检查,代码是否遵守编程语言的规则。若没有检测到错误,则将AST转换为IR,并移交给中端处理。

下面几节中,将为表达式语言构造一个编译器,该编译器将根据其输入生成LLVM IR。LLVM llc静态编译器表示后端,将IR编译成目标代码,这一切都从定义语言开始。切记,本章中所有文件的C++实现,都将包含在src/目录中。