
要使用LLVM,开发系统应该运行一个通用的操作系统,如Linux、FreeBSD、macOS或Windows。可以在不同的模式下构建LLVM和clang,启用调试符号的构建最多需要30 GB的空间。所需的磁盘空间在很大程度上取决于所选择的构建选项。例如,在发布模式下仅构建LLVM核心库,只针对一个平台,需要最少2 GB的可用磁盘空间。

为了减少编译时间,快速的CPU(例如:时钟速度为2.5 GHz的四核CPU)和快速的SSD也很有帮助。甚至可以在小型设备(如Raspberry Pi)上构建LLVM——需要花费很多时间。本书中的示例是在一台笔记本电脑上开发的,该笔记本电脑采用Intel四核CPU,时钟速度为2.7 GHz,具有40gb RAM和2.5TB SSD磁盘空间。

您的开发系统必须安装一些必备软件。让我们回顾一下这些软件包的最低要求版本。

要从GitHub查看源代码,需要Git (\url{https://git-scm.com/})。没有特定版本的要求。GitHub帮助页面建议至少使用1.17.10版本。由于过去发现的已知安全问题,建议使用最新的可用版本,在撰写本文时为2.39.1。

LLVM项目使用CMake(\url{https://cmake.org/})作为构建文件生成器,至少为3.20.0。CMake可以为各种构建系统生成构建文件。在本书中,使用了Ninja(\url{https://ninja-build.org/}),因为它速度快,并且可以在所有平台上使用,建议使用最新版本1.11.1。

显然,还需要一个C/C++编译器。LLVM项目是基于C++17标准,用现代C++编写的。需要一个兼容的编译器和标准库。以下编译器可以与LLVM 17一起工作(已测试):

\begin{itemize}
\item
gcc 7.1.0或更高版本

\item
clang 5.0或更高版本

\item
Apple clang 10.0或更高版本

\item
Visual Studio 2019 16.7或更高版本
\end{itemize}

\begin{myTip}{Tip}
随着LLVM项目的进一步发展,编译器的需求很可能会发生变化。一般来说,应该使用系统可用的最新编译器版本。
\end{myTip}

Python(\url{https://python.org/})用于生成构建文件和运行测试套件,至少为3.8。

虽然本书没有涉及,但可能需要使用Make而不是Ninja。在这种情况下,需要使用GNU Make(\url{https://www.gnu.org/software/make/})3.79或更高版本。这两种构建工具的用法非常相似。对于下面描述的场景,将每个命令中的ninja替换为make就可以了。

LLVM还依赖于zlib库(\url{https://www.zlib.net/}),至少为1.2.3.4版本。与往常一样,建议使用最新版本1.2.13。

要安装必备软件,最简单的方法是从操作系统中使用包管理器。在下面几节中,将为主流操作系统显示安装软件所需的命令。

\mySubsubsection{1.2.1.}{Ubuntu}

Ubuntu 22.04使用apt包管理器。大多数基本的工具都已经安装好了,只缺少开发工具。要一次安装所有软件包,可以输入以下命令:

\begin{shell}
$ sudo apt -y install gcc g++ git cmake ninja-build zlib1g-dev
\end{shell}

\mySubsubsection{1.2.2.}{Fedora和RedHat}

Fedora 37和RedHat Enterprise Linux 9的包管理器名为dnf。和Ubuntu一样,大多数基本的工具都已经安装好了。要一次安装所有软件包,可以输入以下命令:

\begin{shell}
$ sudo dnf –y install gcc gcc-c++ git cmake ninja-build zlib-devel
\end{shell}

\mySubsubsection{1.2.3.}{FreeBSD}

On FreeBSD 13 or later, you have to use the pkg package manager. FreeBSD differs from Linux-based systems in that the clang compiler is already installed. To install all other packages at once, you type the following:

\begin{shell}
$ sudo pkg install –y git cmake ninja zlib-ng
\end{shell}

\mySubsubsection{1.2.4.}{OS X}

在OS X上开发,最好从Apple商店安装Xcode。虽然本书中没有使用Xcode IDE,但它附带了所需的C/C++编译器和相关工具。对于其他工具的安装,可以使用包管理器Homebrew(\url{https://brew.sh/})。要一次安装所有软件包,可以输入以下命令:

\begin{shell}
$ brew install git cmake ninja zlib
\end{shell}

\mySubsubsection{1.2.5.}{Windows}

和OS X一样,Windows没有包管理器。对于C/C++编译器,需要下载个人免费使用的Visual Studio Community 2022(\url{https://visualstudio.microsoft.com/vs/community/})。请确保安装了名为Desktop Development with C++的工作负载。可以使用包管理器Scoop(\url{https://scoop.sh/})来安装其他包。按照网站上的描述安装Scoop之后,从Windows菜单中打开VS 2022的x64 Native Tools Command Prompt。要安装所需的软件包,输入以下命令:

\begin{shell}
$ scoop install git cmake ninja python gzip bzip2 coreutils
$ scoop bucket add extras
$ scoop install zlib
\end{shell}

请密切关注Scoop的输出。对于Python和zlib包,它建议添加一些注册表项。其他软件需要这些条目才能找到这些软件包。要添加注册表项,最好复制并粘贴来自Scoop的输出,如下所示:

\begin{shell}
$ %HOMEPATH%\scoop\apps\python\current\install-pep-514.reg
$ %HOMEPATH%\scoop\apps\zlib\current\register.reg
\end{shell}

在每个命令之后,注册表编辑器将弹出一个消息窗口,询问是否真的要导入这些注册表项,需要单击Yes以完成导入。现在已经安装了所有先决条件。

对于本书中的所有示例,必须在VS 2022中使用x64本机工具命令提示符。使用此命令提示符,编译器将自动添加到搜索路径中。

\begin{myTip}{Tip}
LLVM代码库非常大。为了方便地导航源代码,建议使用一个IDE,允许跳转到类的定义并搜索源代码。我们发现Visual Studio Code(\url{https://code.visualstudio.com/download})是一个可扩展的跨平台IDE,使用起来非常舒服,但这不是运行本书示例的必要条件。
\end{myTip}
