ELF文件格式是LLVM支持的二进制对象文件格式之一。ELF本身为许多CPU架构定义,M88k架构也有定义。我们所需要做的就是,添加重定位的定义和一些标志。重新定位在《System V ABI M88k处理器补充》的第4章,IR代码生成基础(见本章开头设置新后端部分中的链接)中给出:

\begin{enumerate}
\item
我们需要在llvm/include/llvm/BinaryFormat/ELFRelocs/M88k.def文件中进行如下修改:

\begin{cpp}
#ifndef ELF_RELOC
#error "ELF_RELOC must be defined"
#endif
ELF_RELOC(R_88K_NONE, 0)
ELF_RELOC(R_88K_COPY, 1)
// Many more…
\end{cpp}

\item
llvm/include/llvm/BinaryFormat/ELF.h文件中添加了以下标志,以及重定位的定义:

\begin{cpp}
// M88k Specific e_flags
enum : unsigned {
    EF_88K_NABI = 0x80000000, // Not ABI compliant
    EF_88K_M88110 = 0x00000004 // File uses 88110-specific
    features
};

// M88k relocations.
enum {
    #include "ELFRelocs/M88k.def"
};
\end{cpp}

代码可以添加到文件中的任何位置,但最好保持文件的结构,并在MIPS架构代码之前插入。

\item
还需要扩展一些其他的方法。在llvm/include/llvm/Object/ELFObjectFile.h文件中,有一些在枚举成员和字符串之间进行转换的方法,必须在getFileFormatName()方法中添加新的case语句:

\begin{cpp}
switch (EF.getHeader()->e_ident[ELF::EI_CLASS]) {
    // Many more cases
    case ELF::EM_88K:
    return "elf32-m88k";
}
\end{cpp}

\item
类似地,可以扩展getArch()方法:

\begin{cpp}
switch (EF.getHeader().e_machine) {
    // Many more cases
    case ELF::EM_88K:
    return Triple::m88k;
\end{cpp}

\item
最后,在getELFRelocationTypeName()方法中,使用llvm/lib/Object/ELF.cpp文件中的重定位定义:

\begin{cpp}
switch (Machine) {
    // Many more cases
    case ELF::EM_88K:
    switch (Type) {
        #include "llvm/BinaryFormat/ELFRelocs/M88k.def"
        default:
        break;
    }
    break;
}
\end{cpp}

\item
为了完成支持,还可以扩展llvm/lib/ObjectYAML/ELFYAML.cpp文件。该文件由yaml2obj和obj2yaml工具使用,根据YAML描述创建一个ELF文件。第一个添加需要在ScalarEnumerationTraits<ELFYAML::ELF\_EM>::enumeration()方法中完成,该方法列出了所有ELF支持的架构:

\begin{cpp}
ECase(EM_88K);
\end{cpp}

\item
ScalarEnumerationTraits<ELFYAML::ELF\_REL>::enumeration()方法中,需要再次包含重定位的定义:

\begin{cpp}
    case ELF::EM_88K:
#include "llvm/BinaryFormat/ELFRelocs/M88k.def"
    break;
\end{cpp}
\end{enumerate}

至此,已经完成了对ELF文件格式的m88k架构的支持。可以使用llvm-readobj工具检查ELF对象文件,由OpenBSD上的交叉编译器创建的对象文件,可以使用yaml2obj工具为m88k架构创建一个ELF对象文件。

\begin{myNotic}{必须添加对目标文件格式的支持?}
对架构的支持集成到ELF文件格式实现中只需要几行代码,若要为其创建的LLVM后端使用ELF格式,则该采用这种方式。另一方面,添加对全新二进制文件格式的支持是一项复杂的任务。若需要这样做,常用的方法是只输出汇编程序文件,并使用外部汇编程序来创建目标文件。
\end{myNotic}

有了这些补充,ELF文件格式的LLVM实现现在支持M88k架构。下一节中,将创建M88k架构的目标描述,其中描述了该架构的指令、寄存器和更多细节。

















