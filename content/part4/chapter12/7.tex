
通过选择DAG进行指令选择可以生成快速的代码,但是这样做需要时间。对于开发人员来说,编译器的速度至关重要,他们想要快速地尝试他们所做的更改。通常,编译器在优化级别0时应该非常快,但是随着优化级别的增加,可能会花费更多的时间。但构造选择DAG需要花费大量时间,因此这种方法无法按需要进行扩展。第一个解决方案是创建另一个名为FastISel的指令选择算法,该算法速度很快。但不能生成好的代码,也不与选择DAG实现共享代码,这是一个明显的问题,所以并非所有目标都支持FastISel。

选择DAG方法不能扩展,它是一个大型的单片算法。若可以避免创建新的数据结构,如选择DAG,则应该能够使用小组件执行指令选择。后端已经有了一个通道流水线,所以使用通道是很自然的选择。基于这些想法,GlobalISel执行以下步骤:

\begin{enumerate}
\item
首先,将LLVM IR向下转译为通用机器指令,通用机器指令代表了实际硬件中最常见的操作。注意,此转换使用机器函数和机器基本块,所以直接转换为后端其他部分使用的数据结构。

\item
然后,将通用机器指令合法化。

\item
然后,将通用机器指令的操作数映射到注册库。

\item
最后,使用目标描述中定义的模式,将通用指令替换为真正的机器指令。
\end{enumerate}

因为这些都是通道,所以可以在中间插入任意多的通道。例如,组合通道可用于用另一个通用机器指令或真正的机器指令替换通用机器指令序列。关闭这些通道可以提高编译速度,打开它们可以提高生成代码的质量,所以可以根据需要进行扩展。

这种方法还有另一个优点。选择DAG逐个基本块转换基本块,但是机器传递对机器函数工作,这使我们能够在指令选择期间考虑函数的所有基本块,所以这种指令选择方法称为全局指令选择(GlobalISel)。让我们从调用的转换开始,看看这种方法是如何工作的。


\mySubsubsection{12.7.1.}{向下转译参数和返回值}

为了将LLVM IR转换为通用机器指令,只需要实现如何处理参数和返回值。同样,可以通过使用从目标描述生成的代码来向下转译实现。我们要创建的类名为m88kcalllowervm,声明在GISel/M88kcalllowervm .h头文件中:

\begin{cpp}
class M88kCallLowering : public CallLowering {
public:
    M88kCallLowering(const M88kTargetLowering &TLI);

    bool
    lowerReturn(MachineIRBuilder &MIRBuilder,
                const Value *Val,
                ArrayRef<Register> VRegs,
                FunctionLoweringInfo &FLI,
                Register SwiftErrorVReg) const override;

    bool lowerFormalArguments(
        MachineIRBuilder &MIRBuilder, const Function &F,
        ArrayRef<ArrayRef<Register>> VRegs,
        FunctionLoweringInfo &FLI) const override;
    bool enableBigEndian() const override { return true; }
};
\end{cpp}

GlobalISel框架在转换函数时会调用lowerReturn()和lowerFormalArguments()方法。要转换函数调用,还需要覆盖并实现lowerCall()方法。还需要覆写enableBigEndian(),没有它,就会生成错误的机器码。

对于GISel/M88kCallLowering.cpp文件中的实现,需要定义支持类。从目标描述生成的代码告诉我们参数是如何传递的——例如,在寄存器中。需要创建ValueHandler的一个子类来生成机器指令。对于传入参数,需要从IncomingValueHandler派生类,以及从OutgoingValueHandler派生返回值。两者非常相似,所以只查看传入参数的处理程序:

\begin{cpp}
namespace {
struct FormalArgHandler
        : public CallLowering::IncomingValueHandler {
    FormalArgHandler(MachineIRBuilder &MIRBuilder,
                     MachineRegisterInfo &MRI)
        : CallLowering::IncomingValueHandler(MIRBuilder,
                                             MRI) {}

    void assignValueToReg(Register ValVReg,
                          Register PhysReg,
                          CCValAssign VA) override;

    void assignValueToAddress(Register ValVReg,
                              Register Addr, LLT MemTy,
                              MachinePointerInfo &MPO,
                              CCValAssign &VA) override{};

    Register
    getStackAddress(uint64_t Size, int64_t Offset,
                    MachinePointerInfo &MPO,
                    ISD::ArgFlagsTy Flags) override {
        return Register();
    };
};
} // namespace
\end{cpp}

目前,我们只能处理在寄存器中传递的参数,因此必须为其他方法提供一个虚拟实现。assignValueToReg()方法将传入的物理寄存器的值复制到虚拟寄存器,必要时进行截断。我们在这里所要做的,就是将物理寄存器标记为函数的live-in,并调用超类实现:

\begin{cpp}
void FormalArgHandler::assignValueToReg(
        Register ValVReg, Register PhysReg,
        CCValAssign VA) {
    MIRBuilder.getMRI()->addLiveIn(PhysReg);
    MIRBuilder.getMBB().addLiveIn(PhysReg);
    CallLowering::IncomingValueHandler::assignValueToReg(
        ValVReg, PhysReg, VA);
}
\end{cpp}

现在,我们可以实现lowerFormalArgument()方法:

\begin{enumerate}
\item
首先,将IR函数的参数转换为ArgInfo类的实例。setArgFlags()和splitToValueTypes()框架方法有助于,在传入参数需要多个虚拟寄存器的情况下,复制参数属性和分割值类型:

\begin{cpp}
bool M88kCallLowering::lowerFormalArguments(
        MachineIRBuilder &MIRBuilder, const Function &F,
        ArrayRef<ArrayRef<Register>> VRegs,
        FunctionLoweringInfo &FLI) const {
    MachineFunction &MF = MIRBuilder.getMF();
    MachineRegisterInfo &MRI = MF.getRegInfo();
    const auto &DL = F.getParent()->getDataLayout();

    SmallVector<ArgInfo, 8> SplitArgs;
    for (const auto &[I, Arg] :
            llvm::enumerate(F.args())) {
        ArgInfo OrigArg{VRegs[I], Arg.getType(),
                        static_cast<unsigned>(I)};
        setArgFlags(OrigArg,
                    I + AttributeList::FirstArgIndex, DL,
                    F);
        splitToValueTypes(OrigArg, SplitArgs, DL,
                          F.getCallingConv());
    }
\end{cpp}

\item
有了SplitArgs变量中准备的参数,就可以生成机器代码了。在生成的调用约定CC\_M88k和辅助类FormalArghandler帮助下,这一切都由框架代码自行完成:

\begin{cpp}
    IncomingValueAssigner ArgAssigner(CC_M88k);
    FormalArgHandler ArgHandler(MIRBuilder, MRI);
    return determineAndHandleAssignments(
        ArgHandler, ArgAssigner, SplitArgs, MIRBuilder,
        F.getCallingConv(), F.isVarArg());
}
\end{cpp}
\end{enumerate}

返回值的处理类似,主要区别是最多返回一个值。下一个任务是使通用机器指令合法化。

\mySubsubsection{12.7.2.}{通用机器指令合法化}

从LLVM IR到通用机器码的转换大部分是固定的,所以会生成使用不受支持的数据类型的指令,以及其他挑战,合法化通道的任务是定义哪些操作和指令是合法的。有了这些信息,GlobalISel框架试图将指令转化为法律形式。例如,m88k架构只有32位寄存器,因此使用64位值的按位和操作则不合法。但若将64位值拆分为两个32位值,并使用两个按位和操作,就有了合法的代码。这可以转化为一个合法化规则:

\begin{cpp}
getActionDefinitionsBuilder({G_AND, G_OR, G_XOR})
    .legalFor({S32})
    .clampScalar(0, S32, S32);
\end{cpp}

无论何时,当合法化程序处理一条G\_AND指令时,若所有操作数都是32位宽的,就是合法的。否则,操作数将被限制为32位,有效地将较大的值拆分为多个32位值,并再次应用该规则。若一条指令不能合法化,那么后端就会以一条错误消息终止。

所有的合法化规则都在M88kLegalizerInfo类的构造函数中定义,这使得该类非常简单。

\begin{myTip}{合法化是什么意思?}
在GlobalISel中,若通用指令可以让指令选择器翻译,就是合法的。这给了我们在实现上更多的自由。例如,只要指令选择器能够正确处理该类型,就可以声明一条指令处理位值,即使硬件只处理32位值。
\end{myTip}

我们需要看的下一关是寄存器库的选择器。

\mySubsubsection{12.7.3.}{为操作数选择一个寄存器库}

许多架构定义了多个寄存器库,寄存器库是一组寄存器。典型的寄存器库有通用寄存器库和浮点寄存器库。为什么这些信息很重要?在寄存器库中,将值从一个寄存器移动到另一个寄存器通常很便宜,但将值复制到另一个寄存器库可能代价高昂,甚至不可能,所以必须为每个操作数选择一个合适的寄存器库。

该类的实现涉及到对目标描述的添加。在GISel/M88lRegisterbanks.td文件中,定义了一个寄存器库,引用了我们定义的寄存器类:

\begin{cpp}
def GRRegBank : RegisterBank<"GRRB", [GPR, GPR64]>;
\end{cpp}

这一行,生成了一些支持代码,但仍然需要添加一些可能生成的代码。首先,需要定义部分映射。这告诉框架一个值从哪个位索引开始,它有多宽,以及它映射到哪个寄存器库。我们有两个条目,每个寄存器类一个:

\begin{cpp}
RegisterBankInfo::PartialMapping
    M88kGenRegisterBankInfo::PartMappings[]{
        {0, 32, M88k::GRRegBank},
        {0, 64, M88k::GRRegBank},
    };
\end{cpp}

要索引这个数组,必须定义一个枚举:

\begin{cpp}
enum PartialMappingIdx { PMI_GR32 = 0, PMI_GR64, };
\end{cpp}

因为只有三个地址指令,所以需要三个部分映射,每个操作数对应一个。必须用所有这些指针创建一个数组,其中第一个表项表示无效映射:

\begin{cpp}
RegisterBankInfo::ValueMapping
    M88kGenRegisterBankInfo::ValMappings[]{
        {nullptr, 0},
        {&M88kGenRegisterBankInfo::PartMappings[PMI_GR32], 1},
        {&M88kGenRegisterBankInfo::PartMappings[PMI_GR32], 1},
        {&M88kGenRegisterBankInfo::PartMappings[PMI_GR32], 1},
        {&M88kGenRegisterBankInfo::PartMappings[PMI_GR64], 1},
        {&M88kGenRegisterBankInfo::PartMappings[PMI_GR64], 1},
        {&M88kGenRegisterBankInfo::PartMappings[PMI_GR64], 1},
    };
\end{cpp}

要访问该数组,必须定义一个函数:

\begin{cpp}
const RegisterBankInfo::ValueMapping *
M88kGenRegisterBankInfo::getValueMapping(
        PartialMappingIdx RBIdx) {
    return &ValMappings[1 + 3*RBIdx];
}
\end{cpp}

创建这些表时,很容易出错。所有这些信息都可以从目标描述中获得,并且源代码中的注释指出该代码应该由TableGen!然而,这还没有实现,所以必须手动创建代码。

必须在M88kRegisterBankInfo类中实现的最重要的函数是getInstrMapping(),为指令的每个操作数返回映射的寄存器库。现在这变得很简单,可以查找部分映射数组,然后可以将其传递给getInstructionMapping()方法,该方法构造完整的指令映射:

\begin{cpp}
const RegisterBankInfo::InstructionMapping &
M88kRegisterBankInfo::getInstrMapping(
        const MachineInstr &MI) const {
    const ValueMapping *OperandsMapping = nullptr;
    switch (MI.getOpcode()) {
    case TargetOpcode::G_AND:
    case TargetOpcode::G_OR:
    case TargetOpcode::G_XOR:
        OperandsMapping = getValueMapping(PMI_GR32);
        break;
    default:
#if !defined(NDEBUG) || defined(LLVM_ENABLE_DUMP)
        MI.dump();
#endif
        return getInvalidInstructionMapping();
    }

    return getInstructionMapping(DefaultMappingID, /*Cost=*/1,
                                 OperandsMapping,
                                 MI.getNumOperands());
}
\end{cpp}

在开发过程中,通常会忘记通用指令的寄存器库映射。

不幸的是,在运行时生成的错误消息没有提及映射失败的指令。简单的修复方法是在返回无效映射之前转储指令。但这里需要小心,dump()方法并非在所有构建类型中都可用。

在映射寄存器库之后,必须将通用机器指令转换成真正的机器指令。

\mySubsubsection{12.7.4.}{翻译通用机器指令}

对于通过选择DAG进行指令选择,目标描述中添加了使用DAG操作和操作数的模式。为了重用这些模式,引入了从DAG节点类型到通用机器指令的映射。例如,DAG和操作映射到通用的G\_AND机器指令。并非所有DAG操作都具有等效的通用机器指令,但涵盖了最常见的情况。因此,有利于在目标描述中定义所有的代码选择模式。

M88kInstructionSelector类的大部分实现(可以在GISel/M88kInstructionSelector.cpp文件中找到)是从目标描述生成的。然而,需要重写select()方法,可以翻译目标描述中模式未涵盖的通用机器指令。由于只支持非常小的通用指令子集,所以可以简单地调用生成的模式匹配器:

\begin{cpp}
bool M88kInstructionSelector::select(MachineInstr &I) {
    if (selectImpl(I, *CoverageInfo))
        return true;
    return false;
}
\end{cpp}

随着指令选择的实现,可以使用GlobalISel翻译LLVM IR !

\mySubsubsection{12.7.5.}{运行一个示例}

要使用GlobalISel翻译LLVM IR,需要在llc的命令行中添加-global-isel选项。例如,可以使用前面定义的IR文件,and.ll:

\begin{shell}
$ llc -mtriple m88k-openbsd -global-isel < and.ll
\end{shell}

输出的汇编文本相同。为了让自己相信翻译使用了GlobalISel,必须利用这样一个事实,即可以在使用-stop-after=选项运行指定的传递后停止翻译。例如,要查看合法化后的通用指令,可以执行以下命令:

\begin{shell}
$ llc -mtriple m88k-openbsd -global-isel < and.ll -stop-after=legalizer
\end{shell}

因为GlobalISel可使调试和测试实现变得容易,其另一个优点是能够在运行一次通道之后(或之前)停止。

现在,我们有了一个工作后端,可以将一些LLVM IR转换为m88k架构的机器码。让我们考虑一下如何基于此,实现一个更完整的后端。

