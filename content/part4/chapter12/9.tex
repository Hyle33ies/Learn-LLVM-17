本章中,后端添加了两种不同的指令选择:通过选择DAG进行指令选择和全局指令选择,所以必须在目标描述中定义调用约定。此外,需要实现寄存器和指令信息类,能够访问从目标描述生成的信息,但还需要使用其他信息对其进行增强,了解到堆栈帧布局和脚本生成稍后才需要。为了转换示例,添加了一个类来发出机器指令,并创建了后端配置,还了解了全局指令选择的工作原理。最后,获得了一些关于如何自己开发后端的指导。

下一章中,将介绍一些在指令选择之后可以完成的任务——将在后端流水线中添加一个新通道,看看如何将后端集成到clang编译器中,以及如何交叉编译到不同的架构中。

