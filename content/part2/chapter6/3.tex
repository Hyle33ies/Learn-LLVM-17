
为了允许源代码级调试,必须添加调试信息。LLVM中对调试信息的支持使用调试元数据来描述源语言的类型和其他静态信息,并使用intrinsic来跟踪变量值。LLVM核心库在Unix系统上以DWARF格式生成调试信息,在Windows系统上以PDB格式生成调试信息。我们将在下一节中查看其总体结构。

\mySubsubsection{6.3.1.}{理解调试元数据的一般结构}

为了描述通用结构,LLVM使用的元数据类似于用于基于类型分析的元数据。静态结构描述了文件、编译单元、函数和词法块,以及使用的数据类型。

使用的主要类是llvm::DIBuilder,需要使用llvm/IR/DIBuilder头文件来获取类声明。此构建器类提供了一个易于使用的接口来创建调试元数据。之后,元数据将添加到LLVM对象(如全局变量)中,或者在调试内部函数的调用中使用。下面是一些构建器类可以创建的重要元数据:

\begin{itemize}
\item
llvm::DIFile:使用文件名和包含该文件的目录的绝对路径描述一个文件。可以使用createFile()方法来创建,文件可以包含主编译单元,也可以包含导入的声明。

\item
llvm::DICompileUnit:用于描述当前编译单元。除其他事项外,还指定源语言、特定于编译器的生产者字符串、是否启用优化,当然还有DIFile(编译单元驻留在其中)。可以通过createCompileUnit()来创建。

\item
llvm:: dissubprogram:描述一个函数。这里最重要的信息是作用域(通常是嵌套函数的DICompileUnit或dissubprogram)、函数名、修改后的函数名和函数类型。可以通过调用createFunction()来创建。

\item
llvm::DILexicalBlock:描述了一个词法块,并对许多高级语言中的块范围进行了建模。可以通过调用createLexicalBlock()来创建。
\end{itemize}

LLVM对编译器翻译的语言不做任何假设,所以没有关于该语言的数据类型的信息。要支持源代码级调试,特别是在调试器中显示变量值,还必须添加类型信息。下面是一些重要的构念:

\begin{itemize}
\item
createBasicType()函数返回一个指向llvm::DIBasicType类的指针,创建元数据来描述基本类型,例如tinylang中的INTEGER或C++中的int。除了类型的名称之外,所需的参数还包括以位为单位的大小和编码——例如,若是有符号类型或无符号类型。

\item
有几种方法可以构造复合数据类型的元数据,如llvm::DIComposite类所示。可以使用createArrayType()、createStructType()、createUnionType()和createVectorType()函数分别实例化数组、结构、联合和矢量数据类型的元数据。这些函数需要期望的参数,例如基类型和数组类型的订阅数量或结构类型的字段成员列表。

\item
还有支持枚举、模板、类等的方法。
\end{itemize}

函数列表显示必须将源语言的每个细节添加到调试信息中,假设llvm::DIBuilder类的实例称为DBuilder,还假设在一个名为file的文件中有一些tinylang源文件。/home/llvmuser目录下的File.mod文件中,其第5行是Func():INTEGER函数,第7行包含一个局部VAR i:INTEGER声明。让我们为此创建元数据,从文件的信息开始。这里,需要指定文件名和文件所在文件夹的绝对路径:

\begin{cpp}
llvm::DIFile *DbgFile = DBuilder.createFile("File.mod",
                                            "/home/llvmuser");
\end{cpp}

该文件在tinylang中是一个模块,这使得它成为LLVM的编译单元。这包含了很多信息:

\begin{cpp}
bool IsOptimized = false;
llvm::StringRef CUFlags;
unsigned ObjCRunTimeVersion = 0;
llvm::StringRef SplitName;
llvm::DICompileUnit::DebugEmissionKind EmissionKind =
    llvm::DICompileUnit::DebugEmissionKind::FullDebug;
llvm::DICompileUnit *DbgCU = DBuilder.createCompileUnit(
    llvm::dwarf::DW_LANG_Modula2, DbgFile, "tinylang",
    IsOptimized, CUFlags, ObjCRunTimeVersion, SplitName,
    EmissionKind);
\end{cpp}

此外,调试器需要知道源语言。DWARF标准定义了一个包含所有公共值的枚举,这样做的一个缺点是不能简单地添加新的源语言。要做到这一点,必须在DWARF委员会创建一个请求。请注意,调试器和其他调试工具也需要支持一种新语言——仅仅向枚举中添加一个新成员远远不够。

许多情况下,选择一种接近源语言的语言就足够了。在tinylang的例子中,这是Modula-2,使用DW\_LANG\_Modula2作为语言标识符。编译单元保存在文件中,该文件由我们前面创建的DbgFile变量表示。

此外,调试信息可以携带有关生成器的信息,这些信息可以是编译器的名称和版本信息。在这里,只是传递了tinylang字符串。若不想添加这些信息,则可以简单地使用一个空字符串作为参数。

下一组信息包括IsOptimized标志,指示编译器是否开启了优化。通常,这个标志来自于-O命令行开关,可以使用CUFlags参数将其他参数设置传递给调试器。这里没有使用这个,直接传递一个空字符串。我们也不使用Objective-C,所以将0作为Objective-C运行时参数。

通常,调试信息嵌入在我们正在创建的目标文件中。若想要将调试信息写入一个单独的文件,则SplitName参数必须包含这个文件的名称;否则,只需传递一个空字符串就足够了。最后,可以定义应该发出的调试信息的级别。默认值是完整的调试信息,正如FullDebug枚举值的使用所表明的那样,但若只想生成行号,也可以选择LineTablesOnly值,或者根本不生成调试信息的NoDebug值。对于后者,最好一开始就不要创建调试信息。

我们的极简源码只使用INTEGER数据类型,这是一个带符号的32位值。为这种类型创建元数据很简单:

\begin{cpp}
llvm::DIBasicType *DbgIntTy =
    DBuilder.createBasicType("INTEGER", 32,
        llvm::dwarf::DW_ATE_signed);
\end{cpp}

要为函数创建调试元数据,必须首先为签名创建类型,然后为函数本身创建元数据,这类似于为函数创建IR。函数的签名是一个数组,其中所有类型的参数按源顺序排列,函数的返回类型作为索引0处的第一个元素。通常,这个数组是动态构造的。在本例中,还可以静态地构造元数据。这对于内部函数很有用,比如模块初始化。通常,这些函数的参数是已知的,编译器编写器可以硬编码它们:

\begin{cpp}
llvm::Metadata *DbgSigTy = {DbgIntTy};
llvm::DITypeRefArray DbgParamsTy =
            DBuilder.getOrCreateTypeArray(DbgSigTy);
llvm::DISubroutineType *DbgFuncTy =
            DBuilder.createSubroutineType(DbgParamsTy);
\end{cpp}

我们的函数有INTEGER返回类型,没有其他参数,因此DbgSigTy数组只包含指向该类型元数据的指针。该静态数组被转换为类型数组,然后用于创建函数的类型。

函数本身需要更多的数据:

\begin{cpp}
unsigned LineNo = 5;
unsigned ScopeLine = 5;
llvm::DISubprogram *DbgFunc = DBuilder.createFunction(
    DbgCU, "Func", "_t4File4Func", DbgFile, LineNo,
    DbgFuncTy, ScopeLine, llvm::DISubprogram::FlagPrivate,
    llvm::DISubprogram::SPFlagLocalToUnit);
\end{cpp}

函数属于编译单元,在示例中,编译单元存储在DbgCU变量中。需要在源文件中指定函数的名称,即Func,并且修改后的名称存储在目标文件中。这些信息有助于调试器定位函数的机器码。根据tinylang规则,修改后的名称为\_t4File4Func,还必须指定包含函数的文件。

乍一看,这可能令人惊讶,但想想C和C++中的包含机制:函数可以存储在不同的文件中,然后在主编译单元中使用\#include包含该文件。这里的情况并非如此,我们使用与编译单元使用的文件相同的文件。接下来,传递函数的行号和函数类型。函数的行号不能是函数的词法作用域开始的行号,所以可以指定不同的ScopeLine。函数也有保护,我们在这里用FlagPrivate值指定它来指示private函数。函数保护的其他可能值是FlagPublic和FlagProtected,分别用于public和protected域内的函数。

除了保护级别,这里还可以指定其他标志。例如,FlagVirtual表示虚函数,FlagNoReturn表示该函数不返回给调用者。可以在LLVM include文件中找到可能值的完整列表,即llvm/include/llvm/IR/DebugInfoFlags.def。

最后,可以指定特定于函数的标志。最常用的标志是SPFlagLocalToUnit值,表明该函数是编译单元的本地函数。MainSubprogram值也经常使用,表明这个函数是应用程序的主要函数。前面提到的LLVM包含文件,还列出了与特定于函数的标志相关的所有可能值。

到目前为止,我们只创建了引用静态数据的元数据。变量是动态的,因此将在下一节中探讨如何将静态元数据添加到IR代码中以访问变量。

\mySubsubsection{6.3.2.}{跟踪变量及其值}

前一节中描述的类型元数据需要与源程序的变量相关联。对于全局变量,这非常简单。llvm::DIBuilder类的createGlobalVariableExpression()函数创建描述全局变量的元数据。这包括源文件中变量的名称、修改过的名称、源文件等。LLVM IR中的全局变量由GlobalVariable类的实例表示。这个类有一个名为addDebugInfo()的方法,可将从createGlobalVariableExpression()返回的元数据节点与全局变量关联起来。

对于局部变量,需要采取另一种方法。LLVM IR不知道表示局部变量的类,因为只知道值。LLVM社区开发的解决方案是将对函数的调用插入到函数的IR代码中。内在函数是LLVM知道的函数,所以可以用它做一些神奇的事情。在大多数情况下,内在函数不会导致机器级别的子例程调用。在这里,函数调用可将元数据与值关联起来,函数内最重要的调试元数据是llvm.dbg.declare和llvm.dbg.value。

llvm.dbg.declare内部变量提供信息,由前端生成,用于声明局部变量。本质上,this描述了局部变量的地址。在优化过程中,通道可以用(可能多次)调用llvm.dbg来替换这个内在的调用。值来保存调试信息并跟踪本地源变量。在优化之后,llvm.dbg.declare用于描述局部变量在内存中的程序点,所以可能会出现多次调用。

另一方面,只要将局部变量设置为新值,就会调用llvm.dbg.value。这个内部变量描述的是局部变量的值,而不是它的地址。

这一切是如何运作的呢?LLVM IR表示和通过LLVM::DIBuilder类进行的编程创建略有不同,因此我们将对两者进行研究。

继续上一节的示例,我们将使用alloca指令为function函数中的I变量分配本地存储:

\begin{shell}
@i = alloca i32
\end{shell}

之后,必须添加一个对llvm.dbg.declare的调用:

\begin{shell}
call void @llvm.dbg.declare(metadata ptr %i,
                metadata !1, metadata !DIExpression())
\end{shell}

第一个参数是局部变量的地址。第二个参数是描述局部变量的元数据,是通过调用本地变量createAutoVariable()或调用llvm::DIBuilder类参数createParameterVariable()来创建的。最后,第三个参数描述一个地址表达式,稍后将对此进行解释。

让我们实现IR创建,可以通过调用llvm::IRBuilder<>类的CreateAlloca()方法来为本地@i变量分配存储空间:

\begin{cpp}
llvm::Type *IntTy = llvm::Type::getInt32Ty(LLVMCtx);
llvm::Value *Val = Builder.CreateAlloca(IntTy, nullptr, "i");
\end{cpp}

LLVMCtx变量是使用的上下文类,而Builder是llvm::IRBuilder<>类的使用实例。

局部变量也需要通过元数据来描述:

\begin{cpp}
llvm::DILocalVariable *DbgLocalVar =
    Dbuilder.createAutoVariable(DbgFunc, "i", DbgFile,
                                7, DbgIntTy);
\end{cpp}

使用上一节中的值,可以指定该变量是DbgFunc函数的一部分,称为i,在DbgFile文件的第7行中定义,并且是DbgIntTy类型。

最后,使用llvm.dbg.declare内部变量将调试元数据与变量的地址关联起来。使用llvm::DIBuilder避免添加调用的所有细节:

\begin{cpp}
llvm::DILocation *DbgLoc =
                llvm::DILocation::get(LLVMCtx, 7, 5, DbgFunc);
DBuilder.insertDeclare(Val, DbgLocalVar,
                        DBuilder.createExpression(), DbgLoc,
                        Val.getParent());
\end{cpp}

同样,必须为变量指定一个源位置。llvm::DILocation的实例是一个容器,保存与作用域相关联的位置的行和列。此外,insertDeclare()方法将调用添加到LLVM IR的内部函数。就这个函数的参数而言,需要存储在Val中的变量的地址,以及存储在DbgValVar中的变量的调试元数据。还传递了一个空地址表达式和前面创建的调试位置,与普通指令一样,需要指定将调用插入到哪个基本块中。若指定一个基本块,则在末尾插入调用。或者,可以指定一条指令,然后在该指令之前插入调用。还有一个指向alloca指令的指针,这是插入到底层基本块中的最后一条指令。因此,可以使用这个基本块,并且调用会添加在alloca指令之后。

若局部变量的值发生了变化,那么必须向IR添加对llvm.dbg.value的调用,以设置局部变量的新值。可以使用llvm::DIBuilder类的insertValue()方法来实现。

当实现函数的IR生成时,我们使用了一种高级算法,主要使用值,避免为局部变量分配存储。在添加调试信息方面,所以我们使用llvm.dbg.value的频率比在clang生成的IR中看到的要高得多。

若变量没有专用存储空间,而是更大的聚合类型的一部分,该怎么办?可能出现这种情况的一种情况是使用嵌套函数。要实现对调用者堆栈帧的访问,必须收集结构中所有使用的变量,并将指向该记录的指针传递给调用函数。在调用的函数内部,可以引用调用方的变量,就好像是函数的局部变量一样。不同的是,这些变量现在是总量的一部分。

对llvm.dbg.declare的调用中,若调试元数据描述了第一个参数所指向的整个内存,则使用空表达式。但若只描述内存的一部分,则需要添加一个表达式,指示元数据应用于内存的哪一部分。

嵌套帧的情况下,需要计算帧中的偏移量。需要访问DataLayout实例,可以从创建IR代码的LLVM模块获得该实例。若llvm::Module实例命名为Mod,保存嵌套框架结构的变量命名为frame,并且是llvm::StructType类型,可以通过以下方式访问框架的第三个成员。这个访问会提供成员的偏移量:

\begin{cpp}
const llvm::DataLayout &DL = Mod->getDataLayout();
uint64_t Ofs = DL.getStructLayout(Frame)->getElementOffset(3);
\end{cpp}

此外,表达式是由一系列操作创建的。要访问帧的第三个成员,调试器需要向基指针添加偏移量。作为一个例子,需要创建一个数组和相关信息:

\begin{cpp}
llvm::SmallVector<int64_t, 2> AddrOps;
AddrOps.push_back(llvm::dwarf::DW_OP_plus_uconst);
AddrOps.push_back(Offset);
\end{cpp}

从这个数组中,可以创建表达式,然后传递给llvm.dbg.declare,而不是空表达式:

\begin{cpp}
llvm::DIExpression *Expr = DBuilder.createExpression(AddrOps);
\end{cpp}

需要注意的是,用户并不限于使用这种偏移操作。DWARF知道许多不同的运算符,可以创建相当复杂的表达式。可以在LLVM包含文件(llvm/include/llvm/BinaryFormat/Dwarf.def)中找到完整的操作符列表。

此时,可以为变量创建调试信息。为了使调试器能够遵循源代码中的控制流,还需要提供行号信息。这是下一节的主题。

\mySubsubsection{6.3.3.}{添加行号}

调试器允许程序员逐行调试应用程序,所以调试器需要知道哪些机器指令属于源代码中的哪一行,LLVM允许为每条指令添加一个源位置。在前一节中,创建了llvm::DILocation类型的位置信息,调试位置提供的信息不仅仅是行、列和范围。若需要,可以指定该行内联到的作用域。还可以指出此调试位置属于隐式代码——即前端生成,但不在源码中的代码。

在此信息可以附加到指令之前,必须将调试位置包装在llvm::DebugLoc对象中。为此,必须简单地将从llvm::DILocation类获得的位置信息传递给llvm::DebugLoc构造函数。通过这种包装,LLVM可以跟踪位置信息。虽然源代码中的位置不会改变,但可以在优化期间删除为源码级语句或表达式生成的机器码,这种封装有助于处理这些变化。

添加行号信息主要归结为从AST检索行号信息并将其添加到生成的指令中。指令类具有setDebugLoc()方法,该方法将位置信息添加到指令上。

在下一节中,将学习如何生成调试信息,并将其添加到tinylang编译器中。

\mySubsubsection{6.3.4.}{使tinylang支持调试}

我们将调试元数据的生成封装在新的CGDebugInfo类中。此外,我们将声明放在tinylang/CodeGen/CGDebugInfo.h头文件中,并将定义放在tinylang/CodeGen/CGDebugInfo.cpp文件中。

CGDebugInfo类有五个重要的成员,需要一个对模块代码生成器CGM的引用,因为需要将类型从AST表示转换为LLVM类型。当然,还需要一个名为Dbuilder的llvm::DIBuilder类的实例。还需要一个指向编译单元实例的指针,将其存储在成员CU中。

为了避免再次为类型创建调试元数据,必须添加一个映射来缓存此信息。这个成员叫做TypeCache。最后,需要一种方法来管理范围信息,为此必须基于llvm::SmallVector<>类ScopeStack创建一个堆栈:

\begin{cpp}
CGModule &CGM;
llvm::DIBuilder DBuilder;
llvm::DICompileUnit *CU;
llvm::DenseMap<TypeDeclaration *, llvm::DIType *>
    TypeCache;
llvm::SmallVector<llvm::DIScope *, 4> ScopeStack;
\end{cpp}

CGDebugInfo类的以下方法利用了这些成员:

\begin{enumerate}
\item
首先,需要在构造函数中创建编译单元,还在这里创建了包含编译单元的文件。稍后,可以通过CU成员引用该文件。

构造函数的代码如下所示:

\begin{cpp}
CGDebugInfo::CGDebugInfo(CGModule &CGM)
        : CGM(CGM), DBuilder(*CGM.getModule()) {
    llvm::SmallString<128> Path(
        CGM.getASTCtx().getFilename());
    llvm::sys::fs::make_absolute(Path);

    llvm::DIFile *File = DBuilder.createFile(
        llvm::sys::path::filename(Path),
        llvm::sys::path::parent_path(Path));

    bool IsOptimized = false;
    llvm::StringRef CUFlags;
    unsigned ObjCRunTimeVersion = 0;
    llvm::StringRef SplitName;
    llvm::DICompileUnit::DebugEmissionKind EmissionKind =
        llvm::DICompileUnit::DebugEmissionKind::FullDebug;
    CU = DBuilder.createCompileUnit(
        llvm::dwarf::DW_LANG_Modula2, File, "tinylang",
        IsOptimized, CUFlags, ObjCRunTimeVersion,
        SplitName, EmissionKind);
}
\end{cpp}

\item
通常,我们需要提供行号。行号可以从源管理器位置派生,这在大多数AST节点中都是可用的。源代码管理器可以将其转换为行号:

\begin{cpp}
    CGDebugInfo::CGDebugInfo(CGModule &CGM)
unsigned CGDebugInfo::getLineNumber(SMLoc Loc) {
    return CGM.getASTCtx().getSourceMgr().FindLineNumber(
        Loc);
}
\end{cpp}

\item
关于作用域的信息保存在堆栈中,需要打开和关闭作用域以及检索当前作用域的方法。编译单元是全局作用域,则会自动添加:

\begin{cpp}
llvm::DIScope *CGDebugInfo::getScope() {
    if (ScopeStack.empty())
    openScope(CU->getFile());
    return ScopeStack.back();
}

void CGDebugInfo::openScope(llvm::DIScope *Scope) {
    ScopeStack.push_back(Scope);
}

void CGDebugInfo::closeScope() {
    ScopeStack.pop_back();
}
\end{cpp}

\item
接下来,必须为需要转换的类型的每个类别创建一个方法。getPervasiveType()方法为基本类型创建调试元数据。注意encoding参数的使用,将INTEGER类型声明为有符号类型,并将BOOLEAN类型编码为布尔值:

\begin{cpp}
llvm::DIType *
CGDebugInfo::getPervasiveType(TypeDeclaration *Ty) {
    if (Ty->getName() == "INTEGER") {
        return DBuilder.createBasicType(
            Ty->getName(), 64, llvm::dwarf::DW_ATE_signed);
    }
    if (Ty->getName() == "BOOLEAN") {
        return DBuilder.createBasicType(
            Ty->getName(), 1, llvm::dwarf::DW_ATE_boolean);
    }
    llvm::report_fatal_error(
        "Unsupported pervasive type");
}
\end{cpp}

\item
若只是重命名类型名,必须将其映射到类型定义。这里,需要利用作用域和行号信息:

\begin{cpp}
llvm::DIType *
CGDebugInfo::getAliasType(AliasTypeDeclaration *Ty) {
    return DBuilder.createTypedef(
        getType(Ty->getType()), Ty->getName(),
        CU->getFile(), getLineNumber(Ty->getLocation()),
        getScope());
}
\end{cpp}

\item
为数组创建调试信息需要指定数组大小和对齐方式,可以从DataLayout类中检索这些数据,还需要指定数组的索引范围:

\begin{cpp}
llvm::DIType *
CGDebugInfo::getArrayType(ArrayTypeDeclaration *Ty) {
    auto *ATy =
        llvm::cast<llvm::ArrayType>(CGM.convertType(Ty));
    const llvm::DataLayout &DL =
        CGM.getModule()->getDataLayout();

    Expr *Nums = Ty->getNums();
    uint64_t NumElements =
        llvm::cast<IntegerLiteral>(Nums)
            ->getValue()
            .getZExtValue();
    llvm::SmallVector<llvm::Metadata *, 4> Subscripts;
    Subscripts.push_back(
        DBuilder.getOrCreateSubrange(0, NumElements));
    return DBuilder.createArrayType(
        DL.getTypeSizeInBits(ATy) * 8,
        1 << Log2(DL.getABITypeAlign(ATy)),
        getType(Ty->getType()),
        DBuilder.getOrCreateArray(Subscripts));
}
\end{cpp}

\item
使用所有这些单一方法,可以创建一个中心方法来为类型创建元数据。当然,这些元数据还负责缓存数据:

\begin{cpp}
llvm::DIType *
CGDebugInfo::getType(TypeDeclaration *Ty) {
    if (llvm::DIType *T = TypeCache[Ty])
        return T;
    if (llvm::isa<PervasiveTypeDeclaration>(Ty))
        return TypeCache[Ty] = getPervasiveType(Ty);
    else if (auto *AliasTy =
        llvm::dyn_cast<AliasTypeDeclaration>(Ty))
            return TypeCache[Ty] = getAliasType(AliasTy);
    else if (auto *ArrayTy =
        llvm::dyn_cast<ArrayTypeDeclaration>(Ty))
            return TypeCache[Ty] = getArrayType(ArrayTy);
    else if (auto *RecordTy =
        llvm ::dyn_cast<RecordTypeDeclaration>(Ty))
        r   eturn TypeCache[Ty] = getRecordType(RecordTy);
    llvm::report_fatal_error("Unsupported type");
    return nullptr;
}
\end{cpp}

\item
还需要添加一个方法来为全局变量生成元数据:

\begin{cpp}
void CGDebugInfo::emitGlobalVariable(
VariableDeclaration *Decl,
llvm::GlobalVariable *V) {
    llvm::DIGlobalVariableExpression *GV =
        DBuilder.createGlobalVariableExpression(
            getScope(), Decl->getName(), V->getName(),
            CU->getFile(),
            getLineNumber(Decl->getLocation()),
            getType(Decl->getType()), false);
    V->addDebugInfo(GV);
}
\end{cpp}

\item
要为过程发送调试信息,需要为过程类型创建元数据。为此,需要一个参数类型的列表,第一个条目的返回类型。若过程没有返回类型,则必须使用未指定类型;这称为void,与C语言类似。若形参是引用,则需要添加引用类型;否则,必须将类型添加到列表中:

\begin{cpp}
llvm::DISubroutineType *
CGDebugInfo::getType(ProcedureDeclaration *P) {
    llvm::SmallVector<llvm::Metadata *, 4> Types;
    const llvm::DataLayout &DL =
        CGM.getModule()->getDataLayout();
    // Return type at index 0
    if (P->getRetType())
        Types.push_back(getType(P->getRetType()));
    else
        Types.push_back(
            DBuilder.createUnspecifiedType("void"));
    for (const auto *FP : P->getFormalParams()) {
        llvm::DIType *PT = getType(FP->getType());
        if (FP->isVar()) {
            llvm::Type *PTy = CGM.convertType(FP->getType());
            PT = DBuilder.createReferenceType(
                llvm::dwarf::DW_TAG_reference_type, PT,
                DL.getTypeSizeInBits(PTy) * 8,
                1 << Log2(DL.getABITypeAlign(PTy)));
        }
        Types.push_back(PT);
    }
    return DBuilder.createSubroutineType(
        DBuilder.getOrCreateTypeArray(Types));
}
\end{cpp}

\item
对于过程本身,现在可以使用在前一步中创建的过程类型创建调试信息。过程也会打开一个新的作用域,所以必须将过程压入作用域堆栈。这里,还必须将LLVM函数对象与新的调试信息关联起来:

\begin{cpp}
void CGDebugInfo::emitProcedure(
        ProcedureDeclaration *Decl, llvm::Function *Fn) {
    llvm::DISubroutineType *SubT = getType(Decl);
    llvm::DISubprogram *Sub = DBuilder.createFunction(
        getScope(), Decl->getName(), Fn->getName(),
        CU->getFile(), getLineNumber(Decl->getLocation()),
        SubT, getLineNumber(Decl->getLocation()),
        llvm::DINode::FlagPrototyped,
        llvm::DISubprogram::SPFlagDefinition);
    openScope(Sub);
    Fn->setSubprogram(Sub);
}
\end{cpp}

\item
当到达过程的末尾时,必须通知构建器完成该过程的调试信息的构造。我们还需要从作用域堆栈中删除这个过程:

\begin{cpp}
void CGDebugInfo::emitProcedureEnd(
ProcedureDeclaration *Decl, llvm::Function *Fn) {
    if (Fn && Fn->getSubprogram())
        DBuilder.finalizeSubprogram(Fn->getSubprogram());
    closeScope();
}
\end{cpp}

\item
最后,当我们添加完调试信息后,需要在构建器上实现finalize()方法。然后,验证生成的调试信息。这是开发过程中的一个重要步骤,可以帮助找到错生成的元数据:

\begin{cpp}
void CGDebugInfo::finalize() { DBuilder.finalize(); }
\end{cpp}

\end{enumerate}

调试信息应该只在用户请求时生成,所以需要一个新的命令行开关。我们将把它添加到CGModule类的文件中,也将在这个类中使用它:

\begin{cpp}
static llvm::cl::opt<bool>
    Debug("g", llvm::cl::desc("Generate debug information"),
        llvm::cl::init(false));
\end{cpp}

-g选项可以与tinylang编译器一起使用,以生成调试元数据。

此外,CGModule类持有std::unique\_ptr<CGDebugInfo>类的实例。指针在用于设置命令行开关的构造函数中初始化:

\begin{cpp}
    if (Debug)
        DebugInfo.reset(new CGDebugInfo(*this));
\end{cpp}

在CGModule.h中定义的getter方法中,我们只返回指针:

\begin{cpp}
CGDebugInfo *getDbgInfo() {
    return DebugInfo.get();
}
\end{cpp}

生成调试元数据的常用模式是检索指针,并检查它是否有效。例如,在创建一个全局变量之后,可以像这样添加调试信息:

\begin{cpp}
VariableDeclaration *Var = …;
llvm::GlobalVariable *V = …;
if (CGDebugInfo *Dbg = getDbgInfo())
    Dbg->emitGlobalVariable(Var, V);
\end{cpp}

为了添加行号信息,需要在CGDebugInfo类中使用一个名为getDebugLoc()的转换方法,将AST中的位置信息转换为调试元数据:

\begin{cpp}
llvm::DebugLoc CGDebugInfo::getDebugLoc(SMLoc Loc) {
    std::pair<unsigned, unsigned> LineAndCol =
        CGM.getASTCtx().getSourceMgr().getLineAndColumn(Loc);
    llvm::DILocation *DILoc = llvm::DILocation::get(
        CGM.getLLVMCtx(), LineAndCol.first, LineAndCol.second,
        getScope());
    return llvm::DebugLoc(DILoc);
}
\end{cpp}

此外,可以调用CGModule类中的实用函数来将行号信息添加到指令中:

\begin{cpp}
void CGModule::applyLocation(llvm::Instruction *Inst,
                             llvm::SMLoc Loc) {
    if (CGDebugInfo *Dbg = getDbgInfo())
        Inst->setDebugLoc(Dbg->getDebugLoc(Loc));
}
\end{cpp}

通过这种方式,可以添加编译器的调试信息。





































