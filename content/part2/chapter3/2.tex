The project layout for tinylang follows the approach we laid out in Chapter 1, Installing LLVM.
The source code for each component is in a subdirectory of the lib directory, and the header files are in a subdirectory of include/tinylang. The subdirectory is named after the component. In Chapter 1, Installing LLVM, we only created the Basic component.

From the previous chapter, we know that we need to implement a lexer, a parser, an AST, and a semantic analyzer. Each is a component of its own, called Lexer, Parser, AST, and Sema, respectively. The directory layout that will be used in this chapter looks like this:

\myGraphic{0.3}{content/part2/chapter3/images/1.png}{Figure 3.1 – The directory layout of the tinylang project}

The components have clearly defined dependencies. Lexer depends only on Basic. Parser depends on Basic, Lexer, AST, and Sema. Sema only depends on Basic and AST. The welldefined dependencies help us reuse the components.

Let’s have a closer look at the implementation!