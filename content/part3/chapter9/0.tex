
LLVM核心库拥有ExecutionEngine组件,该组件允许在内存中编译和执行中间表示(IR)代码。使用这个组件,可以构建即时(JIT)编译器,其允许直接执行IR代码。因为不需要将目标代码存储在辅助存储器上,所以JIT编译器的工作方式更像解释器。

本章中,将了解JIT编译器的应用,以及LLVM JIT编译器的工作原理。将探索LLVM动态编译器和解释器,如何自己实现JIT编译器工具,如何将JIT编译器作为静态编译器的一部分使用,以及相关的挑战。

本章中,将了解以下内容:

\begin{itemize}
\item
获得LLVM的JIT实现和用例的概述

\item
使用JIT编译直接执行

\item
根据现有类实现自己的JIT编译器

\item
从头开始实现自己的JIT编译器
\end{itemize}

本章结束时,将了解如何开发JIT编译器,可以使用预配置的类,也可以使用适合自己的版本。





















