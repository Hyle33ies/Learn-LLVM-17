
clang静态分析器是一个对C、C++和Objective C源代码执行额外检查的工具。静态分析器执行的检查比编译器执行的检查更彻底。在时间和所需资源方面,成本也更高。静态分析器有一组检查器,用于检查某些错误。

该工具执行源代码的符号解释,通过应用程序查看所有代码路径,并从中派生出应用程序中使用的值的约束。符号解释是编译器中常用的一种技术,例如:用于识别常数值。静态分析器的上下文中,检查器应用于派生值。

例如,若除法的除数为零,则静态分析器就会发出警告。可以在div.c文件中存储下面的例子来进行检查:

\begin{cpp}
int divbyzero(int a, int b) { return a / b; }

int bug() { return divbyzero(5, 0); }
\end{cpp}

本例中,静态分析器将对除0发出警告。但编译时,使用clang -Wall -c div.c命令编译该文件时,将不会显示任何警告。

有两种方法可以从命令行调用静态分析器。旧的工具是scan-build,包含在LLVM中,可以用于简单的场景。新的工具是CodeChecker,可在\url{https://github.com/Ericsson/codechecker/}上获得。要检查单个文件,扫描构建工具是最简单的解决方案。只需将编译命令传递给工具,其他的工作都是会自动完成:

\begin{shell}
$ scan-build clang -c div.c
scan-build: Using '/usr/home/kai/LLVM/llvm-17/bin/clang-17' for static
analysis
div.c:2:12: warning: Division by zero [core.DivideZero]
    return a / b;
             ~~^~~
1 warning generated.
scan-build: Analysis run complete.
scan-build: 1 bug found.
scan-build: Run 'scan-view /tmp/scan-build-2021-03-01-023401-8721-1'
to examine bug reports.
\end{shell}

屏幕上的输出已经说明发现了一个问题,触发DivideZero检查器。在/tmp目录的上述子目录中,可以找到HTML格式的完整报告。可以使用scan-view命令查看报告,或者打开在浏览器的子目录中找到的index.html文件。

报告的第一页展示了发现的漏洞摘要:

\myGraphic{0.7}{content/part3/chapter10/images/3.png}{图10.3 - 摘要页面}

对于发现的每个错误,摘要页面显示了错误的类型、源码中的位置,以及分析器发现错误之后的路径长度。还提供了指向该错误的详细报告的链接。

下面的截图显示了错误的详细报告:

\myGraphic{0.7}{content/part3/chapter10/images/4.png}{图10.4 -详细报告}

有了这个详细的报告,就可以通过跟踪编号的气泡来验证错误。我们的简单示例展示了传递0作为参数值,如何导致除零错误。

因此,需要人工验证。若派生的约束对于某个检查器来说不够精确,则可能出现误报。

不仅限于该工具提供的检查程序——还可以添加新的检查程序。

\mySubsubsection{10.5.1.}{向clang静态分析器添加新的检查器}

许多C库提供了必须成对使用的函数。例如,C标准库提供了malloc()和free()函数。由malloc()函数分配的内存必须由free()函数释放一次。不调用free()函数,或者多次调用是一个编程错误。这种编码模式还有很多实例,静态分析器为其中一些实例提供了检查器。

iconv库提供了将文本从一种编码转换为另一种编码的函数——例如,从Latin-1编码转换为UTF-16编码。要执行转换,实现需要分配内存。为了透明地管理内部资源,iconv库提供了iconv\_open()和iconv\_close()函数,必须成对使用,类似于内存管理函数。没有为这些函数实现检查器,让我们实现一个。

要向clang静态分析器添加一个新的检查器,必须创建一个新的checker类的子类。静态分析器通过代码尝试所有可能的路径。分析器引擎在某些点生成事件——例如,在函数调用之前或之后。若需要处理这些事件,对应的类必须提供回调,Checker类和事件的注册在clang/include/clang/StaticAnalyzer/Core/Checker.h头文件中提供。

通常,检查器需要跟踪一些符号。不知道分析器引擎当前尝试的代码路径,所以检查器无法管理状态。因此,跟踪的状态必须向引擎注册,并且只能使用ProgramStateRef实例进行更改。

为了检测错误,检查器需要跟踪从iconv\_open()函数返回的描述符。分析器引擎为iconv\_open()函数的返回值返回一个SymbolRef实例。我们将这个符号与一个状态关联起来,以反映iconv\_close()是否调用。对于状态,我们创建了IconvState类,其封装了一个bool值。

新的IconvChecker类需要处理四种类型的事件:

\begin{itemize}
\item
PostCall,发生在函数调用之后。调用iconv\_open()函数之后,检索返回值的符号,并记住其处于“打开”状态。

\item
PreCall,发生在函数调用之前。调用iconv\_close()函数之前,检查描述符的符号是否处于“打开”状态。若没有,则已经为描述符调用了iconv\_close()函数,并且已经检测到对该函数的两次调用。

\item
DeadSymbols,在清理未使用的符号时发生。我们检查描述符中未使用的符号是否仍处于“打开”状态。是的话,则检测到缺少对iconv\_close()的调用,这是一个资源泄漏。

\item
PointerEscape,当分析器无法再跟踪符号时调用该函数。这种情况下,因为不能再推断描述符是否已关闭,所以需要从状态中删除符号。
\end{itemize}

可以创建一个新目录来实现新的检查器作为clang插件,并在IconvChecker.cpp文件中添加实现:

\begin{enumerate}
\item
对于实现,需要包含几个头文件。发布报告需要包含文件BugType.h,头文件Checker.h提供了Checker类的声明和事件的回调,它们在CallEvent.h中声明。此外,CallDescription.h文件有助于匹配函数和方法。最后,需要使用CheckerContext.h文件来声明CheckerContext类,该类是提供对分析器状态访问的中心类:

\begin{cpp}
#include "clang/StaticAnalyzer/Core/BugReporter/BugType.h"
#include "clang/StaticAnalyzer/Core/Checker.h"
#include "clang/StaticAnalyzer/Core/PathSensitive/CallDescription.h"
#include "clang/StaticAnalyzer/Core/PathSensitive/CallEvent.h"
#include "clang/StaticAnalyzer/Core/PathSensitive/CheckerContext.h"
#include "clang/StaticAnalyzer/Frontend/CheckerRegistry.h"
#include <optional>
\end{cpp}

\item
为了避免输入命名空间的名称,可以使用clang和ento命名空间:

\begin{cpp}
using namespace clang;
using namespace ento;
\end{cpp}

\item
我们将一个状态与代表图标描述符的每个符号关联起来。状态可以是打开的,也可以是关闭的,使用bool类型的变量,打开状态的值为true。状态值封装在IconvState结构体中,该结构体与FoldingSet数据结构一起使用,FoldingSet是一个过滤重复条目的哈希集。为了使用此数据结构实现,这里添加了Profile()方法,该方法设置该结构的唯一位。将结构体放入匿名命名空间中,以避免污染全局命名空间。这个类提供了getOpened()和getClosed()工厂方法以及isOpen()查询方法,而非对外暴露bool值:

\begin{cpp}
namespace {
class IconvState {
    const bool IsOpen;

    IconvState(bool IsOpen) : IsOpen(IsOpen) {}

public:
    bool isOpen() const { return IsOpen; }

    static IconvState getOpened() {
        return IconvState(true);
    }

    static IconvState getClosed() {
        return IconvState(false);
    }

    bool operator==(const IconvState &O) const {
        return IsOpen == O.IsOpen;
    }

    void Profile(llvm::FoldingSetNodeID &ID) const {
        ID.AddInteger(IsOpen);
    }
};
} // namespace
\end{cpp}

\item
IconvState结构体表示iconv描述符的状态,其由SymbolRef类的符号表示。这最好使用映射来完成,符号作为键,将状态作为值。检查器不能保持状态,儿必须用全局程序状态注册状态,通过REGISTER\_MAP\_WITH\_PROGRAMSTATE宏完成。这个宏引入了IconvStateMap名称,稍后将使用它来访问映射:

\begin{cpp}
REGISTER_MAP_WITH_PROGRAMSTATE(IconvStateMap, SymbolRef,
                               IconvState)
\end{cpp}

\item
我们还在匿名命名空间中实现了IconvChecker类。请求的PostCall, PreCall, DeadSymbols和PointerEscape事件是Checker基类的模板参数:

\begin{cpp}
namespace {
    class IconvChecker
        : public Checker<check::PostCall, check::PreCall,
                         check::DeadSymbols,
                         check::PointerEscape> {
\end{cpp}

\item
IconvChecker类具有CallDescription类型的字段,用于识别程序中对iconv\_open(), iconv()和iconv\_close()的函数调用:

\begin{cpp}
    CallDescription IconvOpenFn, IconvFn, IconvCloseFn;
\end{cpp}

\item
该类还包含对检测到的错误类型的引用:

\begin{cpp}
    std::unique_ptr<BugType> DoubleCloseBugType;
    std::unique_ptr<BugType> LeakBugType;
\end{cpp}

\item
最后,这个类有几个方法。除了构造函数和调用事件的方法之外,还需要一个方法来生成错误报告:

\begin{cpp}
    void report(ArrayRef<SymbolRef> Syms,
                const BugType &Bug, StringRef Desc,
                CheckerContext &C, ExplodedNode *ErrNode,
                std::optional<SourceRange> Range =
                    std::nullopt) const;

public:
    IconvChecker();
    void checkPostCall(const CallEvent &Call,
                       CheckerContext &C) const;
    void checkPreCall(const CallEvent &Call,
                      CheckerContext &C) const;
    void checkDeadSymbols(SymbolReaper &SymReaper,
                          CheckerContext &C) const;
    ProgramStateRef
    checkPointerEscape(ProgramStateRef State,
                       const InvalidatedSymbols &Escaped,
                       const CallEvent *Call,
                       PointerEscapeKind Kind) const;
};
} // namespace
\end{cpp}

\item
IconvChecker类的构造函数的实现,使用函数的名称初始化CallDescription字段,并创建表示错误类型的对象:

\begin{cpp}
IconvChecker::IconvChecker()
    : IconvOpenFn({"iconv_open"}), IconvFn({"iconv"}),
        IconvCloseFn({"iconv_close"}, 1) {
    DoubleCloseBugType.reset(new BugType(
        this, "Double iconv_close", "Iconv API Error"));
    LeakBugType.reset(new BugType(
        this, "Resource Leak", "Iconv API Error",
        /*SuppressOnSink=*/true));
}
\end{cpp}

\item
现在,可以实现第一个调用事件方法checkPostCall(),此方法在分析器执行函数调用后调用。若执行的函数不是一个全局C函数,也没有命名为iconv\_open:

\begin{cpp}
void IconvChecker::checkPostCall(
        const CallEvent &Call, CheckerContext &C) const {
    if (!Call.isGlobalCFunction())
        return;
    if (!IconvOpenFn.matches(Call))
        return;
\end{cpp}

\item
否则,可以尝试以符号的形式获取函数的返回值。为了在全局程序状态中存储具有打开状态的符号,需要从CheckerContext实例中获取一个ProgramStateRef实例。状态是不可变的,将符号添加到状态会产生一个新状态。最后,分析器引擎通过调用addTransition()方法获知新状态:

\begin{cpp}
    if (SymbolRef Handle =
            Call.getReturnValue().getAsSymbol()) {
        ProgramStateRef State = C.getState();
        State = State->set<IconvStateMap>(
            Handle, IconvState::getOpened());
        C.addTransition(State);
    }
}
\end{cpp}

\item
同样,在分析器执行函数之前调用checkPreCall()方法。只对一个名为iconv\_close的全局C函数感兴趣:

\begin{cpp}
void IconvChecker::checkPreCall(
        const CallEvent &Call, CheckerContext &C) const {
    if (!Call.isGlobalCFunction()) {
        return;
    }
    if (!IconvCloseFn.matches(Call)) {
        return;
    }
\end{cpp}

\item
若函数的第一个参数的符号(即iconv描述符)已知,则可以从程序状态中检索该符号的状态:

\begin{cpp}
    if (SymbolRef Handle =
            Call.getArgSVal(0).getAsSymbol()) {
        ProgramStateRef State = C.getState();
        if (const IconvState *St =
                State->get<IconvStateMap>(Handle)) {
\end{cpp}

\item
若状态表示关闭状态,则已经检测到双重关闭错误,并且可以生成错误报告。若已经为这个路径生成了错误报告,则调用generateErrorNode()返回一个nullptr值,所以必须检查这种情况:

\begin{cpp}
            if (!St->isOpen()) {
                if (ExplodedNode *N = C.generateErrorNode()) {
                    report(Handle, *DoubleCloseBugType,
                            "Closing a previous closed iconv "
                            "descriptor",
                            C, N, Call.getSourceRange());
                }
                return;
            }
        }
\end{cpp}

\item
否则,必须将符号的状态设置为“关闭”状态:

\begin{cpp}
        State = State->set<IconvStateMap>(
            Handle, IconvState::getClosed());
        C.addTransition(State);
    }
}
\end{cpp}

\item
调用checkDeadSymbols()方法来清理未使用的符号。我们循环遍历跟踪的所有符号,并询问SymbolReaper实例当前符号是否已无效:

\begin{cpp}
void IconvChecker::checkDeadSymbols(
        SymbolReaper &SymReaper, CheckerContext &C) const {
    ProgramStateRef State = C.getState();
    SmallVector<SymbolRef, 8> LeakedSyms;
    for (auto [Sym, St] : State->get<IconvStateMap>()) {
        if (SymReaper.isDead(Sym)) {
\end{cpp}

\item
若符号已无效,则需要检查状态。若状态仍然是打开的,这是潜在的资源泄漏。有一个例外:iconv\_open()在出现错误时返回-1。若分析器位于处理此错误的代码路径中,则由于函数调用失败而假定资源泄漏是错误的。可以尝试从ConstraintManager实例中获取符号的值,若该值为-1,则不认为符号是资源泄漏。可向SmallVector实例添加一个泄漏符号,以便稍后生成错误报告。最后,从程序状态中移除无效符号:

\begin{cpp}
            if (St.isOpen()) {
                bool IsLeaked = true;
                if (const llvm::APSInt *Val =
                State->getConstraintManager().getSymVal(
                State, Sym))
                IsLeaked = Val->getExtValue() != -1;
                if (IsLeaked)
                LeakedSyms.push_back(Sym);
                }
                State = State->remove<IconvStateMap>(Sym);
            }
        }
\end{cpp}

\item
循环之后,调用generateNonFatalErrorNode(),此方法转换到新的程序状态,若此路径还没有错误节点,则返回一个错误节点。LeakedSyms容器保存了泄漏符号的列表(可能为空),调用report()方法来生成错误报告:

\begin{cpp}
        if (ExplodedNode *N =
                C.generateNonFatalErrorNode(State)) {
            report(LeakedSyms, *LeakBugType,
                "Opened iconv descriptor not closed", C, N);
        }
    }
\end{cpp}

\item
当分析器检测到无法跟踪参数的函数调用时,调用checkPointerEscape()函数,必须假设我们不知道iconv描述符是否会在函数内部关闭。异常是对iconv()的调用,执行转换并且已知不调用iconv\_close()函数,以及iconv\_close()函数本身,会在checkPreCall()方法中处理。若调用是在系统头文件中,并且知道在调用的函数中参数不会转义,则不会改变状态。其他情况下,都会从状态中删除符号:

\begin{cpp}
ProgramStateRef IconvChecker::checkPointerEscape(
        ProgramStateRef State,
        const InvalidatedSymbols &Escaped,
        const CallEvent *Call,
        PointerEscapeKind Kind) const {
    if (Kind == PSK_DirectEscapeOnCall) {
        if (IconvFn.matches(*Call) ||
            IconvCloseFn.matches(*Call))
            return State;
        if (Call->isInSystemHeader() ||
            !Call->argumentsMayEscape())
        return State;
    }
    for (SymbolRef Sym : Escaped)
        State = State->remove<IconvStateMap>(Sym);
    return State;
}
\end{cpp}

\item
report()方法生成一个错误报告。该方法的重要参数是符号数组、错误类型和错误描述。该方法内部,为每个符号创建一个错误报告,并将该符号标记为对该错误感兴趣的符号。若源范围作为参数提供,则会添加到报告中,并生成报告:

\begin{cpp}
void IconvChecker::report(
        ArrayRef<SymbolRef> Syms, const BugType &Bug,
        StringRef Desc, CheckerContext &C,
        ExplodedNode *ErrNode,
        std::optional<SourceRange> Range) const {
    for (SymbolRef Sym : Syms) {
        auto R = std::make_unique<PathSensitiveBugReport>(
            Bug, Desc, ErrNode);
        R->markInteresting(Sym);
        if (Range)
            R->addRange(*Range);
        C.emitReport(std::move(R));
    }
}
\end{cpp}

\item
现在,需要在CheckerRegistry实例中注册新的检查器。当插件加载时,使用clang\_registerCheckers(),我们在其中执行注册。每个检查器都有一个名称,并且属于一个包。因为iconv库是一个标准的POSIX接口,所以可以将IconvChecker称为检查器并将其放入unix打包程序中。这是addChecker()方法的第一个参数,第二个参数是功能的简要文档,第三个参数可以是一个文档的URI,该文档提供有关检查器的更多信息:

\begin{cpp}
extern "C" void
clang_registerCheckers(CheckerRegistry &registry) {
    registry.addChecker<IconvChecker>(
        "unix.IconvChecker",
        "Check handling of iconv functions", "");
}
\end{cpp}

\item
最后,需要声明我们正在使用的静态分析器API的版本,使系统能够确定插件是否兼容:

\begin{cpp}
extern "C" const char clang_analyzerAPIVersionString[] =
    CLANG_ANALYZER_API_VERSION_STRING;
\end{cpp}

这就完成了新检查器的实现。为了构建插件,还需要在CMakeLists.txt文件中创建一个构建描述,该文件与IconvChecker.cpp位于同一目录下:

\item
首先定义所需的CMake版本和项目名称:

\begin{cmake}
cmake_minimum_required(VERSION 3.20.0)
project(iconvchecker)
\end{cmake}

\item
接下来,包括LLVM文件。若CMake不能自动找到文件,则必须设置LLVM\_DIR变量,指向包含CMake文件的LLVM目录:

\begin{cmake}
find_package(LLVM REQUIRED CONFIG)
\end{cmake}

\item
将包含CMake文件的LLVM目录附加到搜索路径中,并从LLVM中包含所需的模块:

\begin{cmake}
list(APPEND CMAKE_MODULE_PATH ${LLVM_DIR})
include(AddLLVM)
include(HandleLLVMOptions)
\end{cmake}

\item
为clang加载CMake定义。若CMake不能自动找到文件,必须指定Clang\_DIR变量,使其指向包含CMake文件的Clang目录:

\begin{cmake}
find_package(Clang REQUIRED)
\end{cmake}

\item
接下来,将包含CMake文件的Clang目录附加到搜索路径中,并从Clang中包含所需的模块:

\begin{cmake}
list(APPEND CMAKE_MODULE_PATH ${Clang_DIR})
include(AddClang)
\end{cmake}

\item
定义头文件和库文件的位置,以及使用哪些定义:

\begin{cmake}
include_directories("${LLVM_INCLUDE_DIR}"
                    "${CLANG_INCLUDE_DIRS}")
add_definitions("${LLVM_DEFINITIONS}")
link_directories("${LLVM_LIBRARY_DIR}")
\end{cmake}

\item
前面的定义设置了构建环境。插入以下命令,定义了插件名称和源文件,并且它是一个clang插件:

\begin{cmake}
add_llvm_library(IconvChecker MODULE IconvChecker.cpp
                 PLUGIN_TOOL clang)
\end{cmake}

\item
Windows上,插件支持与Unix不同,必需的LLVM和clang库必须链接进来。下面的代码确保了这一点:

\begin{cmake}
if(WIN32 OR CYGWIN)
    set(LLVM_LINK_COMPONENTS Support)
    clang_target_link_libraries(IconvChecker PRIVATE
        clangAnalysis
        clangAST
        clangStaticAnalyzerCore
        clangStaticAnalyzerFrontend)
endif()
\end{cmake}
\end{enumerate}

现在,可以配置和构建插件,假设设置了CMAKE\_GENERATOR和CMAKE\_BUILD\_TYPE环境变量:

\begin{shell}
$ cmake -DLLVM_DIR=~/LLVM/llvm-17/lib/cmake/llvm \
        -DClang_DIR=~/LLVM/llvm-17/lib/cmake/clang \
        -B build
$ cmake --build build
\end{shell}

可以使用保存在conf.c文件中的以下源代码来测试新的检查器,该文件对iconv\_close()函数有两次调用:

\begin{cpp}
#include <iconv.h>

void doconv() {
    iconv_t id = iconv_open("Latin1", "UTF-16");
    iconv_close(id);
    iconv_close(id);
}
\end{cpp}

要将插件与扫描-构建脚本一起使用,需要通过-load-plugin选项指定插件的路径。使用conf.c文件运行如下:

\begin{shell}
$ scan-build -load-plugin build/IconvChecker.so clang-17 -c conv.c
scan-build: Using '/home/kai/LLVM/llvm-17/bin/clang-17' for static
analysis
conv.c:6:3: warning: Closing a previous closed iconv descriptor [unix.
IconvChecker]
    6 | iconv_close(id);
      | ^~~~~~~~~~~~~~~
1 warning generated.
scan-build: Analysis run complete.
scan-build: 1 bug found.
scan-build: Run 'scan-view /tmp/scan-build-2023-08-08-114154-12451-1' to examine bug reports.
\end{shell}

至此,已经了解了如何使用自己的检查器扩展clang静态分析器,可以使用这些知识来创建新的通用检查器并将其贡献给社区,或者创建需求专用的构建的检查器,以提高产品的质量。

静态分析器是通过利用clang基础设施构建的。下一节将介绍如何构建自己的clang插件扩展。




























